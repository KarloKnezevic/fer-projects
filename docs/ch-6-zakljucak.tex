\chapter{Zaključak}

Prevođenje jezika nije jednostavan zadatak i kroz vrijeme su se razvile razne tehnike kojima se pokušava
olakšati taj proces. Tako se posao podijelio na podzadatke, no i svaki od njih je zahtjevan za implementirati.

Kod leksičke analize se regularni izrazi moraju prevesti u konačne automate, potrebno je minimizirati 
te automate i simulirati ih. Zbog toga su razvijeni generatori leksičkih analizatora poput jFlexa.

Sintaksnu analizu je još teže implementirati i zato su nastali alati poput CUP-a koji
iz zadane gramatike generiranju LALR metodom LR parser. LALR parseri su jedni od najraširenijih.

Problem sa semantičkom analizom je što ne postoji formalni način da se definiraju semantička 
pravila (za razliku od leksičke i sintaksne analize), no često se u praksi koristi
sintaksnom vođena semantička analiza koja može olakšati implementiranje semantičke analize.

Običaj je da se izvorni program prvo prevede u neki međukod koji je pogodan za optimiranje.
Mi nismo stigli implementirati korak optimiranja (koji je u praksi važan) zbog nedostatka
vremena pa naš prevoditelj direktno prevodi u ciljni jezik.
